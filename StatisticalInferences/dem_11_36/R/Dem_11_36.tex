% Options for packages loaded elsewhere
\PassOptionsToPackage{unicode}{hyperref}
\PassOptionsToPackage{hyphens}{url}
%
\documentclass[
]{article}
\title{Dem 11.36}
\author{Muhammad Reza Fahlevi}
\date{}

\usepackage{amsmath,amssymb}
\usepackage{lmodern}
\usepackage{iftex}
\ifPDFTeX
  \usepackage[T1]{fontenc}
  \usepackage[utf8]{inputenc}
  \usepackage{textcomp} % provide euro and other symbols
\else % if luatex or xetex
  \usepackage{unicode-math}
  \defaultfontfeatures{Scale=MatchLowercase}
  \defaultfontfeatures[\rmfamily]{Ligatures=TeX,Scale=1}
\fi
% Use upquote if available, for straight quotes in verbatim environments
\IfFileExists{upquote.sty}{\usepackage{upquote}}{}
\IfFileExists{microtype.sty}{% use microtype if available
  \usepackage[]{microtype}
  \UseMicrotypeSet[protrusion]{basicmath} % disable protrusion for tt fonts
}{}
\makeatletter
\@ifundefined{KOMAClassName}{% if non-KOMA class
  \IfFileExists{parskip.sty}{%
    \usepackage{parskip}
  }{% else
    \setlength{\parindent}{0pt}
    \setlength{\parskip}{6pt plus 2pt minus 1pt}}
}{% if KOMA class
  \KOMAoptions{parskip=half}}
\makeatother
\usepackage{xcolor}
\IfFileExists{xurl.sty}{\usepackage{xurl}}{} % add URL line breaks if available
\IfFileExists{bookmark.sty}{\usepackage{bookmark}}{\usepackage{hyperref}}
\hypersetup{
  pdftitle={Dem 11.36},
  pdfauthor={Muhammad Reza Fahlevi},
  hidelinks,
  pdfcreator={LaTeX via pandoc}}
\urlstyle{same} % disable monospaced font for URLs
\usepackage[margin=1in]{geometry}
\usepackage{color}
\usepackage{fancyvrb}
\newcommand{\VerbBar}{|}
\newcommand{\VERB}{\Verb[commandchars=\\\{\}]}
\DefineVerbatimEnvironment{Highlighting}{Verbatim}{commandchars=\\\{\}}
% Add ',fontsize=\small' for more characters per line
\usepackage{framed}
\definecolor{shadecolor}{RGB}{248,248,248}
\newenvironment{Shaded}{\begin{snugshade}}{\end{snugshade}}
\newcommand{\AlertTok}[1]{\textcolor[rgb]{0.94,0.16,0.16}{#1}}
\newcommand{\AnnotationTok}[1]{\textcolor[rgb]{0.56,0.35,0.01}{\textbf{\textit{#1}}}}
\newcommand{\AttributeTok}[1]{\textcolor[rgb]{0.77,0.63,0.00}{#1}}
\newcommand{\BaseNTok}[1]{\textcolor[rgb]{0.00,0.00,0.81}{#1}}
\newcommand{\BuiltInTok}[1]{#1}
\newcommand{\CharTok}[1]{\textcolor[rgb]{0.31,0.60,0.02}{#1}}
\newcommand{\CommentTok}[1]{\textcolor[rgb]{0.56,0.35,0.01}{\textit{#1}}}
\newcommand{\CommentVarTok}[1]{\textcolor[rgb]{0.56,0.35,0.01}{\textbf{\textit{#1}}}}
\newcommand{\ConstantTok}[1]{\textcolor[rgb]{0.00,0.00,0.00}{#1}}
\newcommand{\ControlFlowTok}[1]{\textcolor[rgb]{0.13,0.29,0.53}{\textbf{#1}}}
\newcommand{\DataTypeTok}[1]{\textcolor[rgb]{0.13,0.29,0.53}{#1}}
\newcommand{\DecValTok}[1]{\textcolor[rgb]{0.00,0.00,0.81}{#1}}
\newcommand{\DocumentationTok}[1]{\textcolor[rgb]{0.56,0.35,0.01}{\textbf{\textit{#1}}}}
\newcommand{\ErrorTok}[1]{\textcolor[rgb]{0.64,0.00,0.00}{\textbf{#1}}}
\newcommand{\ExtensionTok}[1]{#1}
\newcommand{\FloatTok}[1]{\textcolor[rgb]{0.00,0.00,0.81}{#1}}
\newcommand{\FunctionTok}[1]{\textcolor[rgb]{0.00,0.00,0.00}{#1}}
\newcommand{\ImportTok}[1]{#1}
\newcommand{\InformationTok}[1]{\textcolor[rgb]{0.56,0.35,0.01}{\textbf{\textit{#1}}}}
\newcommand{\KeywordTok}[1]{\textcolor[rgb]{0.13,0.29,0.53}{\textbf{#1}}}
\newcommand{\NormalTok}[1]{#1}
\newcommand{\OperatorTok}[1]{\textcolor[rgb]{0.81,0.36,0.00}{\textbf{#1}}}
\newcommand{\OtherTok}[1]{\textcolor[rgb]{0.56,0.35,0.01}{#1}}
\newcommand{\PreprocessorTok}[1]{\textcolor[rgb]{0.56,0.35,0.01}{\textit{#1}}}
\newcommand{\RegionMarkerTok}[1]{#1}
\newcommand{\SpecialCharTok}[1]{\textcolor[rgb]{0.00,0.00,0.00}{#1}}
\newcommand{\SpecialStringTok}[1]{\textcolor[rgb]{0.31,0.60,0.02}{#1}}
\newcommand{\StringTok}[1]{\textcolor[rgb]{0.31,0.60,0.02}{#1}}
\newcommand{\VariableTok}[1]{\textcolor[rgb]{0.00,0.00,0.00}{#1}}
\newcommand{\VerbatimStringTok}[1]{\textcolor[rgb]{0.31,0.60,0.02}{#1}}
\newcommand{\WarningTok}[1]{\textcolor[rgb]{0.56,0.35,0.01}{\textbf{\textit{#1}}}}
\usepackage{graphicx}
\makeatletter
\def\maxwidth{\ifdim\Gin@nat@width>\linewidth\linewidth\else\Gin@nat@width\fi}
\def\maxheight{\ifdim\Gin@nat@height>\textheight\textheight\else\Gin@nat@height\fi}
\makeatother
% Scale images if necessary, so that they will not overflow the page
% margins by default, and it is still possible to overwrite the defaults
% using explicit options in \includegraphics[width, height, ...]{}
\setkeys{Gin}{width=\maxwidth,height=\maxheight,keepaspectratio}
% Set default figure placement to htbp
\makeatletter
\def\fps@figure{htbp}
\makeatother
\setlength{\emergencystretch}{3em} % prevent overfull lines
\providecommand{\tightlist}{%
  \setlength{\itemsep}{0pt}\setlength{\parskip}{0pt}}
\setcounter{secnumdepth}{-\maxdimen} % remove section numbering
\ifLuaTeX
  \usepackage{selnolig}  % disable illegal ligatures
\fi

\begin{document}
\maketitle

{
\setcounter{tocdepth}{2}
\tableofcontents
}
\begin{center}\rule{0.5\linewidth}{0.5pt}\end{center}

\hypertarget{problems}{%
\section{Problems}\label{problems}}

The dataset consists of variable relating to blood pressure of 15
Peruvians (\(n = 15\)) who have moved from rural, high- altitude areas
to urban, lower altitude areas. The variables in this data sets are:
Systolic blood pressure (\(Y\)), weight (\(X_1\)), height (\(X_2\)), and
pulse.

\begin{verbatim}
##    Weight_kg Heigh_mm Pulse_per_minute Systolic_pressure_mmHg
## 1       71.0     1629               88                    170
## 2       56.5     1569               64                    120
## 3       56.0     1561               68                    125
## 4       61.0     1619               52                    148
## 5       65.0     1566               72                    140
## 6       62.0     1639               72                    106
## 7       53.0     1494               64                    120
## 8       53.0     1568               80                    108
## 9       65.0     1540               76                    124
## 10      57.0     1530               60                    134
## 11      66.5     1622               68                    116
## 12      59.1     1486               72                    114
## 13      64.0     1578               88                    130
## 14      69.5     1645               60                    118
## 15      64.0     1648               60                    138
\end{verbatim}

\begin{enumerate}
\def\labelenumi{\roman{enumi}.}
\tightlist
\item
  Determine if weight and systolic blood pressure are in a linear
  relationship, that is, test whether \(H_0 : \beta_{1.0} = 0\), where
  \(\beta_1\) is the slope of the regressor variable.
\item
  Perform a lack-of-fit test to determine if linear relationship between
  weight and systolic blood pressure is adequate. Draw conclusions.
\item
  Determine if pulse rate influences systolic blood pressure in a linear
  relationship. Which regressor variable is the better predictor of the
  systolic blood pressure?
\end{enumerate}

\hypertarget{demonstrandum}{%
\section{Demonstrandum}\label{demonstrandum}}

Let

\[
x_1 &\stackrel{def} = \text{weight} \\
x_2 &\stackrel{def} = \text{height} \\
p &\stackrel{def} = \text{pulse}
\] The simple linear regression for given data
\(\{(x_i, y_i) : i = 1, 2, \ldots, n\}\) is defined as

\[
Y = \beta_0 + \beta_1x
\] \(Y\) is estimated by \[
\hat{y} = b_0 + b_1x
\] where \(b_0\) and \(b_1\) are regression's coefficient estimator for
\(\beta_0\) and \(\beta_1\), respectively. These estimator is computed
as follows,

\$\$

\begin{aligned}
b_0 = \frac{n\sum_{i = 1}^{n}x_iy_i - \sum_{i = 1}^{n}x_i\sum_{i = 1}^{n}y_i}
{n\sum_{i = 1}^{n}x_i^2 - \left(\sum_{i = 1}^{n}x_i\right)^2}, && \text{and} \\

b_1 = \frac{\sum_{i = 1}^{n}y_i - b_1\sum_{i = 1}^{n}x_i}{n}
\end{aligned}

\$\$

\hypertarget{take-a-glimpse-to-the-data}{%
\subsection{Take a Glimpse to the
Data}\label{take-a-glimpse-to-the-data}}

The following output is the summarize of the given data

\begin{verbatim}
##    Weight_kg        Heigh_mm    Pulse_per_minute Systolic_pressure_mmHg
##  Min.   :53.00   Min.   :1486   Min.   :52.0     Min.   :106.0         
##  1st Qu.:56.75   1st Qu.:1550   1st Qu.:62.0     1st Qu.:117.0         
##  Median :62.00   Median :1569   Median :68.0     Median :124.0         
##  Mean   :61.51   Mean   :1580   Mean   :69.6     Mean   :127.4         
##  3rd Qu.:65.00   3rd Qu.:1626   3rd Qu.:74.0     3rd Qu.:136.0         
##  Max.   :71.00   Max.   :1648   Max.   :88.0     Max.   :170.0
\end{verbatim}

\hypertarget{hypothesis-testing-on-the-slope-for-regressor-variable-x_1}{%
\subsection{\texorpdfstring{Hypothesis Testing on the Slope for
Regressor Variable
\(x_1\)}{Hypothesis Testing on the Slope for Regressor Variable x\_1}}\label{hypothesis-testing-on-the-slope-for-regressor-variable-x_1}}

The hypothesis that's being tested is the slope of the regression line
\(\hat{y} = \beta_0 + \beta_1x\)

\[
\begin{cases}
H_0:\beta_{1.0} = 0 \\
H_1:\beta_{1.1} \neq 0 \\
\end{cases}
\]

In order to make decision with regards to the hypothesis, the analysis
of variances is performed.

\textbf{Step 1.} Construct the linear model for \(X_1\),

\begin{verbatim}
## 
## Call:
## lm(formula = "Systolic_pressure_mmHg~Weight_kg", data = dem_11_36)
## 
## Coefficients:
## (Intercept)    Weight_kg  
##      44.398        1.349
\end{verbatim}

then for \(\hat{y} = b_0 + b_1x_1\), \[
\hat{y} = 44.398 + 1.349x_1
\]

\textbf{Step 2.} Compute the \emph{one-way ANOVA}

\begin{Shaded}
\begin{Highlighting}[]
\FunctionTok{anova}\NormalTok{(lmodels\_X1)}
\end{Highlighting}
\end{Shaded}

\begin{verbatim}
## Analysis of Variance Table
## 
## Response: Systolic_pressure_mmHg
##           Df  Sum Sq Mean Sq F value Pr(>F)  
## Weight_kg  1  805.86  805.86  3.3645 0.0896 .
## Residuals 13 3113.74  239.52                 
## ---
## Signif. codes:  0 '***' 0.001 '**' 0.01 '*' 0.05 '.' 0.1 ' ' 1
\end{verbatim}

\textbf{Step 3.} Conclusion. Let \(\alpha = 0.05\), the critical value
for \(f_\alpha(1, n-2)\)

\begin{verbatim}
## [1] 4.667193
\end{verbatim}

According to \emph{one-way ANOVA} tables, the computed f-values
\(f_\text{reg} = 3.3645\). Hence, \(f_\text{reg} < f_{0.05}(1, 13)\).
Therefore, the hypothesis testing lead to \emph{do not reject
\(H_0 : \beta_{1.0} = 0\) at \(\alpha=0.05\) level of significance}.

\hypertarget{anova-for-testing-linearity-of-regression-y-sim-x_1}{%
\subsection{\texorpdfstring{ANOVA for Testing Linearity of Regression
\(Y \sim x_1\)}{ANOVA for Testing Linearity of Regression Y \textbackslash sim x\_1}}\label{anova-for-testing-linearity-of-regression-y-sim-x_1}}

In order to determine the linear relationship between weight (\(x_1\))
and the systolic blood pressure (\(Y\)) is adequate or not, the
\emph{ANOVA for testing linearity of regression} is performed.

\includegraphics{Dem_11_36_files/figure-latex/unnamed-chunk-8-1.pdf}

\textbf{Step 1.} Compute the sum of square
\(\text{SSR, SSE, SSE(pure)}\) and lack-of-fit. The \(\text{SSR}\) is
computed as follows, \[
\begin{aligned}
\text{SSR} &= \sum_{i = 1}^{n}(\hat{y_i} - \bar{y})^2 \\
&= 805.86
\end{aligned}
\]

on \(1\) degrees of freedom. The \(\text{SSE}\) is computed as follows,
\[
\begin{aligned}
\text{SSE} &= \sum_{i = 1}^{n}(y_i - \hat{y_i})^2 \\
&= 3113.74
\end{aligned}
\]

on \(n-2=13\) degrees of freedom. For data which contain \(k\)-groups,
the \(\text{SSE(pure)}\) is computed as follows, \[
\begin{aligned}
\text{pure error } &:= 
\sum_{i = 1}^{k}\sum_{j = 1}^n (y^{(i)}_j - \bar{y}^{(i)})^2
\end{aligned}
\] where

\[
\bar{y}^{(i)} = \frac{1}{n_i}\sum_{j = 1}^{n_i}y^{(i)}_j
\]

for \(i = 1, 2, \ldots, k\) and \(j = 1, 2, \ldots, n_i\). In order to
compute the pure error, the data must be group by the regressor variable
\((x_1)\).

\begin{verbatim}
## # A tibble: 15 x 2
## # Groups:   Weight_kg [12]
##    Weight_kg Systolic_pressure_mmHg
##        <dbl>                  <dbl>
##  1      53                      120
##  2      53                      108
##  3      56                      125
##  4      56.5                    120
##  5      57                      134
##  6      59.1                    114
##  7      61                      148
##  8      62                      106
##  9      64                      130
## 10      64                      138
## 11      65                      140
## 12      65                      124
## 13      66.5                    116
## 14      69.5                    118
## 15      71                      170
\end{verbatim}

then

\begin{verbatim}
## # A tibble: 12 x 3
##    Weight_kg mean_y_ith sum_sqr_y_ith
##        <dbl>      <dbl>         <dbl>
##  1      53          114            72
##  2      56          125             0
##  3      56.5        120             0
##  4      57          134             0
##  5      59.1        114             0
##  6      61          148             0
##  7      62          106             0
##  8      64          134            32
##  9      65          132           128
## 10      66.5        116             0
## 11      69.5        118             0
## 12      71          170             0
\end{verbatim}

from the tables, \(k = 12\) groups, the second column is equals to
\(\bar{y}^{(i)}\), the third column is the value of
\[\sum_{j = 1}^{n_i}(y^{(i)}_j - \bar{y}^{(i)})\] and the sum of third
column is equals to the so called \(\text{pure error}\), thus, \[
\text{pure error} := 232.00
\] on \(n - k = 3\) degrees of freedom.

The differences between \(\text{SSE}\) and \(\text{SSE(pure)}\) is
equals to the so called \emph{lack-of-fit}, hence, \[
\begin{aligned}
\text{lack-of-fit} := \text{SSE} - \text{SSE(pure)} &=
\sum_{i = 1}^{n}(y_i - \hat{y}_i)^2 -
\sum_{i = 1}^k \sum_{j = 1}^{n_i}(y^{(i)}_j - \bar{y}^{(i)})^2 \\
&= 3113.74 - 232.00 \\
&= 2881.74
\end{aligned}
\] on \(k - 2 = 10\) degrees of freedom.

\textbf{Step 2.} Compute the mean square for
\(\text{SSE}, \text{SSE(pure)}\) and \(\text{lack-of-fit}\). The mean
square for \(\text{SSE}\), \[
\begin{aligned}
s^2 &= \frac{\text{SSE}}{n - 2} = 
\sum_{i = 1}^{n} \frac{(y_i - \hat{y}_i)^2}{n - 2} =
\frac{S_{yy} - b_1S_{xy}}{n - 2} \\
&= \frac{3113.74}{15 - 2} = \frac{3113.74}{13} \\
s^2 &= 239.52
\end{aligned}
\]

The mean square for \emph{pure-error}, \[
\begin{aligned}
s^2_{\text{pure}} &= \frac{\text{pure error}}{n - k} =
\sum_{i = 1}^{k}\sum_{j = 1}^{n_i} \frac{(y^{(i)}_j - 
\bar{y}^{(i)})^2}{n - k} \\
&= \frac{232.00}{15 - 12} \\ &= \frac{232.00}{3} \\
s^2_{\text{pure}} &= 77.33
\end{aligned}
\]

The mean square for \emph{lack-of-fit}, \[
\begin{aligned}
s^2_{\text{lack-of-fit}} &= \frac{\text{lack-of-fit}}{k - 2}
= \frac{\text{SSE} - \text{SSE(pure)}}{k - 2} \\
&= \frac{1}{k - 2} \left\{
\sum_{i = 1}^{n}(y_i - \hat{y}_i)^2 -
\sum_{i = 1}^{k} \sum_{j = 1}^{n_i}(y^{(i)}_j - \bar{y}^{(i)})^2
\right\} \\
&= \frac{3113.74 - 232.00}{12 - 2} \\
&= \frac{2881.74}{10} \\
s^2_{\text{lack-of-fit}} &= 288.17
\end{aligned}
\]

\textbf{Step 3.} Compute the f-values. For \(f_\text{reg}\) \[
\begin{aligned}
f_\text{reg} &= \frac{\text{SSE}}{s^2_\text{pure}} \\
&= \frac{3113.74}{77.33} \\
f_\text{reg} &= 10.4206
\end{aligned}
\]

on 1 and \(n - 2 = 13\) degrees of freedom, and for
\(f_\text{lack-of-fit}\), \[
\begin{aligned}
f_\text{lack-of-fit} &= \frac{\text{lack-of-fit}}{s^2_\text{pure}(k - 2)} \\
&= \frac{\text{SSE} - \text{SSE(pure)}}{s^2_\text{pure}(k - 2)} \\
&= \frac{2881.74}{77.33 \times (12 - 2)} \\
&= \frac{2881.74}{77.33 \times 10} \\
&= \frac{2881.74}{773.3} \\
f_\text{lack-of-fit} &= 3.7264
\end{aligned}
\] on \(k - 2 = 10\) and \(n - k = 3\) degrees of freedom.

\textbf{Step 4.} Compute the P-values

\begin{verbatim}
## [1] "P-values for f_regression := 0.00660075471987764"
\end{verbatim}

\begin{verbatim}
## [1] "P-values for f_lack-of-fit :=  0.153221390704174"
\end{verbatim}

\textbf{Step 5.} Summarize altogether computation as table of
\emph{ANOVA for testing for linearity of regression.}

\begin{Shaded}
\begin{Highlighting}[]
\NormalTok{EnvStats}\SpecialCharTok{::}\FunctionTok{anovaPE}\NormalTok{(lmodels\_X1)}
\end{Highlighting}
\end{Shaded}

\begin{verbatim}
##                      Df  Sum Sq Mean Sq F value  Pr(>F)  
## Weight_kg             1  805.86  805.86 10.4206 0.04829 *
## Lack of Fit          10 2881.74  288.17  3.7264 0.15322  
## Pure Error            3  232.00   77.33                  
## ---
## Signif. codes:  0 '***' 0.001 '**' 0.01 '*' 0.05 '.' 0.1 ' ' 1
\end{verbatim}

\textbf{Step 6 (Conclusion).} Let \(\alpha = 0.05\), then the critical
value for \[f_\alpha(1, n-2) = f_\alpha(1, 13) = 4.667193\], and
\[f_\alpha(k-2, n-k) = f_\alpha(10, 3) = 8.785525\] From the table of
\emph{ANOVA for testing the linearity of regression},
\(f_\text{reg} > f_\alpha(1,13)\) is \textbf{true}, and
\(f_\text{lack-of-fit} > f_\alpha(10, 3)\) is \textbf{false}. Therefore,
there are significant amount of variation accounted for by linear model
(\emph{reject \(H_0 : \beta_{1.0} = 0\)}) and insignificant amount due
to lack of fit. Thus, the experimental data do not seem to suggest the
need to consider terms higher than first order in the model, and the null
hypothesis is not rejected.

\hypertarget{anova-for-testing-linearity-of-regression-y-sim-p}{%
\subsection{\texorpdfstring{ANOVA for Testing Linearity of Regression
\(Y \sim p\)}{ANOVA for Testing Linearity of Regression Y \textbackslash sim p}}\label{anova-for-testing-linearity-of-regression-y-sim-p}}

As usual, in order to determine the linear relationship between pulse
rate (\(p\)) and the systolic blood pressure (\(Y\)) is adequate or not,
the \emph{Analysis of Variances for testing linearity of regression} is
performed.

\textbf{Step 1.} Construct the linear model \(\hat{y} = b_0 + b_1p\)

\begin{verbatim}
## 
## Call:
## lm(formula = "Systolic_pressure_mmHg~Pulse_per_minute", data = dem_11_36)
## 
## Coefficients:
##      (Intercept)  Pulse_per_minute  
##         117.0641            0.1485
\end{verbatim}

therefore, \[\hat{y} = 117.0641 + 0.1485p\]

\includegraphics{Dem_11_36_files/figure-latex/unnamed-chunk-14-1.pdf}

\textbf{Step 2.} Compute the \(\text{SSR}\), \(\text{SSE}\),
\(\text{pure-error}\), and \(\text{lack-of-fit}\).

\textbf{Step 3.} Compute the mean square error for \(\text{SSE}\),
\(\text{pure-error}\), and \(\text{lack-of-fit}\).

\textbf{Step 4.} Compute the \(f\)-value for regression and
\(\text{lack-of-fit}\).

\textbf{Step 5.} Compute the \(P\text{-values}\) for \(f_{\text{reg}}\)
and \(f_{\text{lack-of-fit}}\)

\textbf{Step 6.} Summarize altogether results into table of \emph{ANOVA
for testing linearity of regression}.

\begin{Shaded}
\begin{Highlighting}[]
\NormalTok{EnvStats}\SpecialCharTok{::}\FunctionTok{anovaPE}\NormalTok{(lmodels\_pulse)}
\end{Highlighting}
\end{Shaded}

\begin{verbatim}
##                             Df  Sum Sq Mean Sq F value Pr(>F)
## Pulse_per_minute             1   33.03   33.03  0.1363 0.7229
## Lack of Fit                  6 2190.07  365.01  1.5061 0.3007
## Pure Error                   7 1696.50  242.36
\end{verbatim}

\textbf{Step 7 (Conclusion).} Let \(\alpha = 0.05\), recall the critical
value for \(f_\alpha(1, 13)\) and \(f_\alpha(10, 3)\). From the table of
ANOVA for testing the linearity of regression,
\(f_\text{reg} > f_\alpha(1, 13)\) is \textbf{false}, and
\(f_{\text{lack-of-fit}} > f_\alpha(10, 3)\) is \textbf{false}.
Therefore, there are insignificant amount of variation accounted for by
linear model (\emph{do not reject \(H_0:\beta_{1.0} = 0\)}) and
insignificant amount due to lack of fit. Thus, the experimental data do
not seem to suggest the need to consider terms higher than the first
order.

\hypertarget{hypothesis-testing-on-the-slope-for-regressor-variable-x_2}{%
\subsection{\texorpdfstring{Hypothesis Testing on the Slope for
Regressor Variable
\(x_2\)}{Hypothesis Testing on the Slope for Regressor Variable x\_2}}\label{hypothesis-testing-on-the-slope-for-regressor-variable-x_2}}

The hypothesis that's being tested is the slope of the regression line
\(\hat{y} = \beta_0 + \beta_1x_2\), such that \[
\begin{cases}
H_0:\beta_{1.0} = 0 \\
H_1:\beta_{1.1} \neq 0
\end{cases}
\]

in order decide which hypothesis should be chosen, the \emph{one-way
analysis of variances} is used.

\textbf{Step 1.} Construct the linear model \(\hat{y} = b_0 + b_1x_2\).

\begin{verbatim}
## 
## Call:
## lm(formula = "Systolic_pressure_mmHg~Heigh_mm", data = dem_11_36)
## 
## Coefficients:
## (Intercept)     Heigh_mm  
##     0.80328      0.08014
\end{verbatim}

therefore, \[\hat{y} = 0.80328 + 0.08014x_2\]
\includegraphics{Dem_11_36_files/figure-latex/unnamed-chunk-17-1.pdf}

\textbf{Step 2.} Compute the \(\text{SSR}\) and \(\text{SSE}\).

\textbf{Step 3.} Compute the \(f\text{-values}\).

\textbf{Step 4.} Compute the \(P\text{-values}\).

\textbf{Step 5.} Summarize altogether computation as table of
\emph{One-way ANOVA}.

\begin{verbatim}
## Analysis of Variance Table
## 
## Response: Systolic_pressure_mmHg
##           Df Sum Sq Mean Sq F value Pr(>F)
## Heigh_mm   1  251.6  251.61  0.8917 0.3622
## Residuals 13 3668.0  282.15
\end{verbatim}

\textbf{Step 6 (Conclusion).} Let \(\alpha = 0.05\), recal the critical
value for \(f_\alpha(1, n - 2)\) or \(f_\alpha(1, 13)\). From the table
of one-way ANOVA, \(f\text{-values} > f_\alpha(1,13)\) is
\textbf{false}. Therefore, there are insignificant amount of variation
accounted for by linear model (\emph{do not reject}
\(H_0 : \beta_{1.0} = 0\)).

\hypertarget{determine-which-regressor-variable-is-the-better-predictor-of-y}{%
\subsection{\texorpdfstring{Determine Which Regressor Variable is the
Better Predictor of
\(Y\)}{Determine Which Regressor Variable is the Better Predictor of Y}}\label{determine-which-regressor-variable-is-the-better-predictor-of-y}}

Visualize altogether linear models
\includegraphics{Dem_11_36_files/figure-latex/unnamed-chunk-19-1.pdf}
\includegraphics{Dem_11_36_files/figure-latex/unnamed-chunk-19-2.pdf}
\includegraphics{Dem_11_36_files/figure-latex/unnamed-chunk-19-3.pdf}

With regards to which regressor variable is better, from the computed
\(P\text{-values}\) after performed ANOVA for variable \(x_1, x_2\) and
\(p\), for regressor variable \(x\), \(P\text{-values} \approx 0\), then
\(x\) is a better regressor variable for \(Y\). Therefore, variable
weight \((x_1)\) is better regressor variable for predictor of the
systolic blood pressure \((Y)\), with \(P\text{-values} = 0.04829\).

\end{document}
